% Options for packages loaded elsewhere
\PassOptionsToPackage{unicode}{hyperref}
\PassOptionsToPackage{hyphens}{url}
%
\documentclass[
]{article}
\usepackage{amsmath,amssymb}
\usepackage{iftex}
\ifPDFTeX
  \usepackage[T1]{fontenc}
  \usepackage[utf8]{inputenc}
  \usepackage{textcomp} % provide euro and other symbols
\else % if luatex or xetex
  \usepackage{unicode-math} % this also loads fontspec
  \defaultfontfeatures{Scale=MatchLowercase}
  \defaultfontfeatures[\rmfamily]{Ligatures=TeX,Scale=1}
\fi
\usepackage{lmodern}
\ifPDFTeX\else
  % xetex/luatex font selection
\fi
% Use upquote if available, for straight quotes in verbatim environments
\IfFileExists{upquote.sty}{\usepackage{upquote}}{}
\IfFileExists{microtype.sty}{% use microtype if available
  \usepackage[]{microtype}
  \UseMicrotypeSet[protrusion]{basicmath} % disable protrusion for tt fonts
}{}
\makeatletter
\@ifundefined{KOMAClassName}{% if non-KOMA class
  \IfFileExists{parskip.sty}{%
    \usepackage{parskip}
  }{% else
    \setlength{\parindent}{0pt}
    \setlength{\parskip}{6pt plus 2pt minus 1pt}}
}{% if KOMA class
  \KOMAoptions{parskip=half}}
\makeatother
\usepackage{xcolor}
\usepackage[margin=1in]{geometry}
\usepackage{graphicx}
\makeatletter
\def\maxwidth{\ifdim\Gin@nat@width>\linewidth\linewidth\else\Gin@nat@width\fi}
\def\maxheight{\ifdim\Gin@nat@height>\textheight\textheight\else\Gin@nat@height\fi}
\makeatother
% Scale images if necessary, so that they will not overflow the page
% margins by default, and it is still possible to overwrite the defaults
% using explicit options in \includegraphics[width, height, ...]{}
\setkeys{Gin}{width=\maxwidth,height=\maxheight,keepaspectratio}
% Set default figure placement to htbp
\makeatletter
\def\fps@figure{htbp}
\makeatother
\setlength{\emergencystretch}{3em} % prevent overfull lines
\providecommand{\tightlist}{%
  \setlength{\itemsep}{0pt}\setlength{\parskip}{0pt}}
\setcounter{secnumdepth}{-\maxdimen} % remove section numbering
\ifLuaTeX
  \usepackage{selnolig}  % disable illegal ligatures
\fi
\usepackage{bookmark}
\IfFileExists{xurl.sty}{\usepackage{xurl}}{} % add URL line breaks if available
\urlstyle{same}
\hypersetup{
  pdftitle={MyFitnessPal User LTV Model (2024)},
  pdfauthor={Shoryoung Gong},
  hidelinks,
  pdfcreator={LaTeX via pandoc}}

\title{MyFitnessPal User LTV Model (2024)}
\author{Shoryoung Gong}
\date{2025-07-05}

\begin{document}
\maketitle

\subsection{Paying/Free Users}\label{payingfree-users}

\begin{itemize}
\tightlist
\item
  \textbf{Total MyFitnessPal Users(2024)}: 220 Million\footnote{ElectroIQ,
    2024}
\item
  \textbf{Subscription Conversion Rate}: Based on freemium subscription
  conversion rates from comparable health and fitness apps: Strava
  (2\%)\footnote{Sacra, 2024}, Noom (3.75\%), and Healthify Me (2\%),the
  average subscription conversion rate is approximately \textbf{2.58\%},
  which serves as a benchmark for estimating MyFitnessPal's subscription
  user base.
\item
  \textbf{Annual Revenue(2024)}: \$310 Million\footnote{Businessofapps,2024}
\end{itemize}

\begin{verbatim}
## Step 1: Paying users = 0.0258 * 220 = 5.676 million
\end{verbatim}

\begin{verbatim}
## Step 2: Free users = 220 - 5.676 = 214.324 million
\end{verbatim}

\textbf{Methodology}: Using average subscription conversion rate based
off of relevant freemium subscription health and fitness companies and
multiplying it by the total users to obtain total subscribing users.
Then subtracting that from the total users to find amount of free users.

\subsection{Monetization}\label{monetization}

\begin{itemize}
\tightlist
\item
  \textbf{Subscription}: Approximately 75\% of health and fitness app
  revenue comes from subscriptions\footnote{Businessofapps,2024}.
  HealthifyMe generates 80\%\footnote{Oyelabs,2024} of its revenue from
  subscriptions and Strava generates 90\%\footnote{Sacra, 2024} from
  premium subscriptions. Averaging out the data from similar freemium
  subscription health/fitness apps as well as overall generalizations we
  approximate \textbf{81.6\%} of MyfitnessPal's revenue comes from its
  premium subscriptions.
\item
  \textbf{Advertisements}: Premium subscriptions allows for viewers to
  not receive ads meaning they become mutually exclusive in
  calculations.
\end{itemize}

\begin{verbatim}
## Step 1: Subscription revenue = $310M Annual Revenue * 0.816 = 252.96 million
\end{verbatim}

\begin{verbatim}
## Step 2: Advertisement revenue = $310M Annual Revenue - 252.96 = 57.04 million
\end{verbatim}

\textbf{Methodology}: Using average percentage of revenue from
subscription through relevant freemium subscription health and fitness
companies to multiply by total annual revenue to recieve subscription
and advertisement monetization details.

\subsection{Churning}\label{churning}

\begin{itemize}
\tightlist
\item
  \textbf{Retention}: Industry Standards for free users:
  3.5\%-8\%\footnote{AppsFlyer,2025}. Strava's 30 day retention rate is
  around 19\%\footnote{Alchemer,2021}. HealthifyMe's monthly retention
  rate is around 25\%. The churn and retention rates are highly variable
  across different sites citing industry standard. Additionally,
  MyFitnessPal has recorded a retention rate of 24\% after 90
  days\footnote{MobileAction,2023}.
\end{itemize}

\begin{verbatim}
## Step 1: Estimate churn from 90-day retention = 1 - (0.24)^(1/3) = 0.3786
\end{verbatim}

\begin{verbatim}
## Step 2: Estimated monthly churn = 37.86 %
\end{verbatim}

\begin{verbatim}
## Step 3: Estimated user lifetime = 1 / 0.3786 = 2.64 months
\end{verbatim}

\textbf{Methodology}: R(t) = (1 - c)\^{}t. Where c is the retention rate
and t is the time. We are posturing a monthly churn rate so t=3. Churn
rate is typically measured by 1-(retention rate). Expected lifetime is
calculated by dividing 1 by the estimated monthly churn.

\subsection{LTV Calculation}\label{ltv-calculation}

\textbf{LTV}: Lifetime Value estimates how much revenue a business can
expect from a single user over their lifetime before churning.

The Lifetime Value is calculated as:

\[
\text{LTV} = \text{ARPU} \times \text{Customer Lifetime}
\]

Where:

\begin{itemize}
\tightlist
\item
  \(\text{ARPU}\) = Average Revenue Per User (monthly or annual)\\
  \[
  \text{ARPU} = \frac{\text{Total Revenue}}{\text{Total Users}} = \frac{310M}{220M}
  \] \[
  \text{LTV} = 0.1174*2.64=0.309936
  \] \textbf{Interpretation}: The life time value per MyFitnessPal user
  until they churn is approximately \$0.31.
\end{itemize}

\subsection{Sources}\label{sources}

\begin{itemize}
\tightlist
\item
  Alchemer. (2021). \emph{Strava Retention Metrics}.\\
\item
  AppsFlyer. (2025). \emph{Retention Rates Report}.\\
\item
  Businessofapps. (2024). \emph{MyFitnessPal Revenue and
  Monetization}.\\
\item
  ElectroIQ. (2024). \emph{MyFitnessPal Statistics}.\\
\item
  MobileAction. (2023). \emph{MyFitnessPal Retention Data}.\\
\item
  Oyelabs. (2024). \emph{HealthifyMe Revenue Breakdown}.\\
\item
  Sacra, J. (2024). \emph{Subscription Conversion Rates in Fitness
  Apps}.
\end{itemize}

\end{document}
